% =============================================================================
%  Kapitel: Interaktives Auswertungswerkzeug (Interactive Plotter)
%  Belegarbeit – Raumakustik / Digitale Signalverarbeitung
% =============================================================================
\documentclass[12pt, a4paper]{article}

\usepackage[utf8]{inputenc}
\usepackage[T1]{fontenc}
\usepackage[ngerman]{babel}
\usepackage{lmodern}
\usepackage{microtype}

% Mathematik
\usepackage{amsmath}
\usepackage{amssymb}

% Tabellen
\usepackage{booktabs}
\usepackage{array}
\usepackage{tabularx}

% Code-Listings
\usepackage{listings}
\usepackage{xcolor}

% Layout
\usepackage[
  left=3cm, right=2.5cm, top=2.5cm, bottom=2.5cm
]{geometry}
\usepackage{setspace}
\usepackage{parskip}
\setlength{\parindent}{0pt}
\setlength{\parskip}{6pt}

% Floats & Captions
\usepackage{caption}
\usepackage{float}

% Hyperlinks (letztes Paket)
\usepackage[hidelinks]{hyperref}

% =============================================================================
%  MATLAB-Listing-Stil
% =============================================================================
\definecolor{matlabkw}{rgb}{0.00, 0.00, 0.60}
\definecolor{matlabstr}{rgb}{0.63, 0.13, 0.94}
\definecolor{matlabcmt}{rgb}{0.13, 0.54, 0.13}
\definecolor{matlabbg}{rgb}{0.97, 0.97, 0.97}
\definecolor{matlabnr}{rgb}{0.50, 0.50, 0.50}

\lstdefinestyle{matlab}{
  language        = Matlab,
  basicstyle      = \ttfamily\footnotesize\linespread{1.0}\selectfont,
  keywordstyle    = \color{matlabkw}\bfseries,
  stringstyle     = \color{matlabstr},
  commentstyle    = \color{matlabcmt}\itshape,
  backgroundcolor = \color{matlabbg},
  frame           = single,
  framesep        = 4pt,
  rulecolor       = \color{black!25},
  numbers         = left,
  numberstyle     = \tiny\color{matlabnr},
  numbersep       = 8pt,
  stepnumber      = 1,
  tabsize         = 4,
  breaklines      = true,
  breakatwhitespace = true,
  showstringspaces = false,
  captionpos      = b,
  aboveskip       = 8pt,
  belowskip       = 4pt,
}
\lstset{style=matlab}

% =============================================================================
%  Dokument
% =============================================================================
\begin{document}

% ---------------------------------------------------------------------------
\section{Interaktives Auswertungswerkzeug}
\label{sec:plotter}
% ---------------------------------------------------------------------------

Zur visuellen Inspektion, Plausibilitätsprüfung und vergleichenden Analyse
der verarbeiteten Messdaten wurde ein interaktives GUI-Werkzeug in MATLAB
implementiert (\texttt{interactive\_plotter.m}).
Das Werkzeug operiert auf den durch die Verarbeitungspipeline erzeugten
\texttt{Proc\_*.mat}-Dateien, kann aber auch direkt auf Rohdaten
aus \texttt{dataraw/} zugreifen und diese on-the-fly verarbeiten.

% ---------------------------------------------------------------------------
\subsection{Architektur und Initialisierung}
% ---------------------------------------------------------------------------

Die GUI basiert auf klassischen MATLAB-\texttt{uicontrol}-Elementen innerhalb
einer einzelnen \texttt{figure}. Sämtliche Callback-Funktionen sind als
verschachtelte Funktionen realisiert, die auf gemeinsame Zustandsvariablen
im übergeordneten Workspace zugreifen -- insbesondere Achsengrenzen,
den aktuellen Darstellungstyp und einen Heatmap-Cache.

\begin{lstlisting}[caption={GUI-Grundstruktur und Zustandsverwaltung},
                   label=lst:gui_init]
f = figure('Name', 'Akustik Auswertung', 'NumberTitle', 'off', ...
           'Position', [100, 100, 1200, 700], 'Color', 'w');
pnlControl = uipanel(f, 'Position', [0.01 0.01 0.25 0.98], ...
                      'Title', 'Einstellungen');

% Zustandsvariablen
lastPlotType = 1;
ySettings = containers.Map('KeyType','double','ValueType','any');
ySettings(1) = [-30, 10];    % Spektrum
ySettings(2) = [-1.1, 1.1];  % Impulsantwort
ySettings(3) = [-60, 5];     % ETC
ySettings(4) = [-60, 5];     % EDC
ySettings(5) = [-30, 20];    % Pegel vs. Entfernung
ySettings(6) = [-30, 20];    % 3D Scatter
ySettings(7) = [-60, 0];     % Heatmap
ySettings(8) = [0, 0.4];     % RT60
\end{lstlisting}

Beim Start wird die globale Vollaussteuerungsreferenz $FS_\mathrm{global}$
aus der ersten verfügbaren \texttt{Proc\_*.mat}-Datei geladen.
Rohdaten, die über das Werkzeug geöffnet werden, werden gegen dieselbe
Referenz skaliert, um die Vergleichbarkeit zu gewährleisten:

\begin{lstlisting}[caption={Laden der globalen Referenz beim Start},
                   label=lst:gui_ref]
FS_global_ref = 1.0;
procFiles = dir(fullfile(procDir, 'Proc_*.mat'));
if ~isempty(procFiles)
    tmpLoad = load(fullfile(procDir, procFiles(1).name), 'Result');
    if isfield(tmpLoad.Result.meta, 'FS_global_used')
        FS_global_ref = tmpLoad.Result.meta.FS_global_used;
    end
end
\end{lstlisting}

% ---------------------------------------------------------------------------
\subsection{Betriebsmodi}
% ---------------------------------------------------------------------------

Das Werkzeug bietet zwei Betriebsmodi: \emph{Einzelansicht} und
\emph{Vergleich (Differenz)}.

Im \textbf{Einzelmodus} wird eine ausgewählte Messung dargestellt.
Alle acht Darstellungstypen (s.\,Abschnitt~\ref{sec:darstellungen}) stehen
zur Verfügung. Die Dateiliste wird dynamisch aus dem gewählten
Quellverzeichnis (\texttt{processed/} oder \texttt{dataraw/}) erzeugt.

Im \textbf{Vergleichsmodus} werden zwei Messungen überlagert.
Beim Frequenzspektrum wird zusätzlich ein Differenz-Subplot erzeugt,
der bandweise die Pegeldifferenz $\Delta L_k = L_{2,k} - L_{1,k}$
als Balkendiagramm darstellt:

\begin{lstlisting}[caption={Spektralvergleich mit Differenz-Subplot},
                   label=lst:gui_compare]
if isCompare
    % Oberer Plot: Beide Spektren ueberlagert
    axes(ax1);
    stairs(x_plot, y1_plot, 'b-', 'LineWidth', 1.5, ...
           'DisplayName', name1_leg);
    stairs(x_plot, y2_plot, 'r-', 'LineWidth', 1.5, ...
           'DisplayName', name2_leg);

    % Unterer Plot: Differenz (Messung 2 - Referenz)
    axes(ax2);
    diff_y = y2_sub - y1_sub;
    bar(x_ticks, diff_y, 'FaceColor', [0.5 0.5 0.5], 'BarWidth', 1);
    ylabel('Differenz [dB]');
    title('Differenz (Messung 2 - Messung 1)');
end
\end{lstlisting}

% ---------------------------------------------------------------------------
\subsection{Darstellungstypen}
\label{sec:darstellungen}
% ---------------------------------------------------------------------------

Acht Darstellungstypen decken die wesentlichen akustischen Analyseperspektiven ab.
Die Umschaltung erfolgt über ein Dropdown-Menü; beim Wechsel werden die
jeweiligen Achsengrenzen automatisch gespeichert und wiederhergestellt.

\begin{table}[H]
\centering
\caption{Verfügbare Darstellungstypen}
\label{tab:plottypes}
\begin{tabular}{@{}clll@{}}
\toprule
\textbf{Nr.} & \textbf{Bezeichnung} & \textbf{X-Achse} & \textbf{Y-Achse} \\
\midrule
1 & Frequenzspektrum (Terz) & Frequenz [Hz] & Pegel [dBFS] \\
2 & Impulsantwort (Zeit)    & Zeit [ms]     & Amplitude \\
3 & Energie-Zeit-Kurve (ETC)& Zeit [ms]     & Pegel [dB] \\
4 & Energy Decay Curve (EDC)& Zeit [ms]     & Pegel [dB] \\
5 & Pegel über Entfernung   & Entfernung [m]& Summenpegel [dBFS] \\
6 & 3D~Scatter (Raum)       & X, Y [m]      & Pegel [dBFS] \\
7 & Heatmap (Raumzeit)      & Rasterposition& Pegel [dBFS] \\
8 & Nachhallzeit (RT60)     & Frequenz [Hz] & $T_{30}$ [s] \\
\bottomrule
\end{tabular}
\end{table}

\subsubsection{Frequenzspektrum (Terz)}

Die Terzbandpegel werden als Treppenfunktion (\texttt{stairs}) dargestellt.
Optional können eine A-Bewertung nach IEC~61672 sowie ein Overlay
der theoretischen Luftdämpfung auf einer Sekundärachse zugeschaltet werden.
Die A-Bewertungskurve wird frequenzabhängig berechnet:

\begin{lstlisting}[caption={A-Bewertungsfilter nach IEC~61672},
                   label=lst:gui_aweight]
f2_aw = f_sub.^2;
RA = (12200^2 .* f_sub.^4) ./ ((f2_aw + 20.6^2) .* ...
     sqrt((f2_aw + 107.7^2) .* (f2_aw + 737.9^2)) .* ...
     (f2_aw + 12200^2));
A_dB = 20*log10(RA) + 2.00;
y1_sub = y1_sub + A_dB;   % Korrektur auf dBFS(A)
\end{lstlisting}

\subsubsection{Impulsantwort und Energie-Zeit-Kurve}

Die Zeitdarstellung zeigt die truncierte Impulsantwort in Millisekunden.
Die ETC wird aus dem logarithmierten Absolutwert berechnet:

\begin{equation}
    \mathrm{ETC}(t) = 20 \cdot \log_{10}\!\bigl(|h(t)| + \varepsilon\bigr)
    \label{eq:etc}
\end{equation}

Über eine Checkbox kann zwischen relativer Zeit (ab Onset) und absoluter
Probenzeit (bezogen auf den Original-Datensatz) umgeschaltet werden:

\begin{lstlisting}[caption={Umschaltung relative/absolute Zeitachse},
                   label=lst:gui_abstime]
offset1 = 0;
if hShowAbsTime.Value && isfield(R1.time.metrics, 'idx_start')
    offset1 = (R1.time.metrics.idx_start - 1) / R1.meta.fs * 1000;
end
t1 = (0:length(R1.time.ir)-1) / R1.meta.fs * 1000 + offset1;
\end{lstlisting}

\subsubsection{Energy Decay Curve (EDC)}

Die EDC basiert auf der Schroeder-Rückwärtsintegration
(vgl.\,Gleichung~\ref{eq:edc} im Pipeline-Kapitel) und wird über
\texttt{calc\_edc()} berechnet.
Im Vergleichsmodus werden beide Kurven überlagert, sodass
Unterschiede im Abklingverhalten direkt sichtbar sind.

\subsubsection{Pegel über Entfernung und 3D-Scatter}

Für die entfernungsabhängige Darstellung werden alle Positionen einer Variante
geladen und deren Summenpegel gegen den Quell-Empfänger-Abstand aufgetragen.
Zusätzlich wird eine theoretische $1/r$-Abfallkurve über
\texttt{calc\_ideal\_curve()} eingeblendet.
Der 3D-Scatter projiziert dieselben Daten auf die räumlichen Koordinaten
des Messrasters:

\begin{lstlisting}[caption={Sammlung der Pegelwerte aller Positionen einer Variante},
                   label=lst:gui_dist]
files = dir(fullfile(procDir, sprintf('Proc_%s_Pos*.mat', variante)));
for i = 1:length(files)
    D = load(fullfile(files(i).folder, files(i).name), 'Result');
    val = D.Result.freq.sum_level;
    posNum = str2double(D.Result.meta.position);
    idx = find([geo.pos] == posNum);
    if ~isempty(idx)
        d = geo(idx).distance;
        if d > 0         % Quellposition ausschliessen
            dist(end+1)    = d;
            levels(end+1)  = val;
        end
    end
end
\end{lstlisting}

\subsubsection{Heatmap (Raumzeit)}

Die Heatmap bildet die momentane Energieverteilung auf dem Messraster
zu einem frei wählbaren Zeitpunkt ab.
Ein Schieberegler steuert den Zeitpunkt; eine Animations\-funktion
durchfährt den zeitlichen Verlauf automatisch.
Die Berechnung des Heatmap-Gitters erfolgt über
\texttt{calc\_heatmap\_grid()}, das aus den IRs aller Positionen
die momentane Energie zum gewünschten Zeitpunkt extrahiert und auf
ein $4{\times}4$-Raster abbildet:

\begin{lstlisting}[caption={Heatmap-Daten und Animation},
                   label=lst:gui_heatmap]
t_ms = get(hSliderTime, 'Value');
data1 = get_heatmap_data(R1.meta.variante, 1);  % gecachte IRs
grid1 = calc_heatmap_grid(data1, t_ms, R1.meta.fs, FS_global_ref);

min_db = str2double(get(hEditThreshold, 'String'));
cLim   = [min_db, 0];                           % dBFS-Farbskala
\end{lstlisting}

Die Farbskala verläuft von Weiß (niedriger Pegel) zu Dunkelblau
(hoher Pegel). Der untere Schwellwert ist über ein Eingabefeld einstellbar.
In jeder Zelle wird die Positionsbezeichnung sowie der gerundete Momentanpegel
eingeblendet.

\subsubsection{Nachhallzeit (RT60) über Frequenz}

Die frequenzabhängige Nachhallzeit $T_{30}$ wird als Treppenfunktion pro
Terzband dargestellt.
Zusätzlich zur Einzelmessung wird automatisch der Mittelwert
über alle Positionen derselben Variante berechnet und als gestrichelte
Referenzlinie eingeblendet:

\begin{lstlisting}[caption={Mittelwertbildung der RT60 über alle Positionen},
                   label=lst:gui_rt60avg]
files = dir(fullfile(procDir, sprintf('Proc_%s_Pos*.mat', variante)));
sum_t30   = zeros(size(f_vec));
count_t30 = zeros(size(f_vec));

for i = 1:length(files)
    D = load(fullfile(files(i).folder, files(i).name), 'Result');
    [vals, ~] = calc_rt60_spectrum(D.Result.time.ir, D.Result.meta.fs);
    mask = ~isnan(vals);
    sum_t30(mask)   = sum_t30(mask)   + vals(mask);
    count_t30(mask) = count_t30(mask) + 1;
end
avg_vals = sum_t30 ./ count_t30;
\end{lstlisting}

% ---------------------------------------------------------------------------
\subsection{Datenanbindung und On-the-fly-Verarbeitung}
% ---------------------------------------------------------------------------

Die Ladefunktion \texttt{loadData()} unterscheidet zwischen vorverarbeiteten
und rohen Daten. Vorverarbeitete Dateien enthalten bereits das vollständige
\texttt{Result}-Struct; bei Rohdaten wird die Verarbeitungskette
(Signalauslese, DC-Entfernung, Spektralanalyse) innerhalb des Plotters
durchgeführt:

\begin{lstlisting}[caption={Datenladen: vorverarbeitet vs.\,Rohdaten},
                   label=lst:gui_load]
if sourceType == 1   % Processed
    tmp = load(fullfile(procDir, filename));
    R   = tmp.Result;
else                  % Raw
    [S, meta] = load_and_parse_file(fullfile(dataDir, filename));
    ir = extract_ir(S);
    ir = process_ir_modifications(ir, 'RemoveDC', true, ...
                                       'AutoSave', false);
    R.time.ir = ir;
    R.meta    = meta;
    R.meta.fs = 500e3;
    [L_terz, L_sum, f_center] = calc_terz_spectrum( ...
        ir, R.meta.fs, FS_global_ref, dist, T_val, LF_val);
    R.freq.f_center   = f_center;
    R.freq.terz_dbfs  = L_terz;
    R.freq.sum_level  = L_sum;
end
\end{lstlisting}

% ---------------------------------------------------------------------------
\subsection{Bedienelemente und Exportfunktion}
% ---------------------------------------------------------------------------

Das Steuerungspanel auf der linken Seite der GUI umfasst folgende Elemente:
Moduswahl (Einzel/Vergleich), Datenquellenwahl (vorverarbeitet/roh),
Dateilisten für beide Messkanäle, Darstellungstyp-Auswahl, Frequenzfilter
($4\,\mathrm{kHz}$--$63\,\mathrm{kHz}$), fixierbare X- und Y-Achsen\-grenzen,
sowie darstellungsspezifische Optionen (A-Bewertung, Luftdämpfungs-Overlay,
Energiemodus, absolute Zeitachse).

Die Achsengrenzen werden pro Darstellungstyp in einer \texttt{containers.Map}
gespeichert und beim Wechsel automatisch restauriert.
Dies verhindert, dass z.\,B.\,die Amplituden\-grenzen einer Impulsantwort
auf ein Frequenzspektrum übertragen werden.

Der Button \emph{Plot speichern} exportiert die aktuelle Darstellung
wahlweise als PNG (300\,dpi), PDF oder MATLAB-\texttt{.fig}-Datei.
Dazu wird der Plotbereich in eine temporäre, unsichtbare \texttt{figure}
kopiert und über \texttt{exportgraphics()} gespeichert:

\begin{lstlisting}[caption={Plot-Export in verschiedene Formate},
                   label=lst:gui_save]
f_temp = figure('Visible', 'off', 'Color', 'w', ...
                'Position', [0 0 1000 700]);
new_ax = copyobj(findobj(f, 'Type', 'axes'), f_temp);
exportgraphics(f_temp, savePath, 'Resolution', 300);
delete(f_temp);
\end{lstlisting}

\end{document}
